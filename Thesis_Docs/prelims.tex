\beforepreface

%\afterpage{\blankpage}
%\blankpage
% plagiarism

\prefacesection{Plagiarism statement}

\vskip 10pc \noindent I declare that this thesis is my
own work, except where acknowledged, and has not been submitted for
academic credit elsewhere. 

\vskip 2pc  \noindent I acknowledge that the assessor of this
thesis may, for the purpose of assessing it:
\begin{itemize}
\item Reproduce it and provide a copy to another member of the University; and/or,
\item Communicate a copy of it to a plagiarism checking service (which may then retain a copy of it on its database for the purpose of future plagiarism checking).
\end{itemize}

\vskip 2pc \noindent I certify that I have read and understood the University Rules in
respect of Student Academic Misconduct, and am aware of any potential plagiarism penalties which may 
apply.\vspace{24pt}

\vskip 2pc \noindent By signing this declaration I am agreeing to the statements and conditions above.
\vskip 2pc
Signed: \rule{7cm}{0.25pt} \hfill Date: \rule{4cm}{0.25pt}

%\afterpage{\blankpage}
\afterpage{\cleardoublepage}

% Acknowledgements are optional


\prefacesection{Acknowledgements}

%{\bigskip}By far the greatest thanks must go to my supervisor for
%the guidance, care and support they provided. 
%
%{\bigskip\noindent}Thanks must also go to Emily, Michelle, John and Alex who helped by
%proof-reading the document in the final stages of preparation.
%
%{\bigskip\noindent}Although I have not lived with them for a number of years, my family also deserve many thanks for their encouragement.
%
%{\bigskip\noindent} Thanks go to Robert Taggart for allowing his thesis style to be shamelessly copied.
%
%{\bigskip\bigskip\bigskip\noindent} Fred Flintstone, 2 November 2015.

%\afterpage{\blankpage}
\afterpage{\cleardoublepage}

% Abstract

\prefacesection{Abstract}

Cerebral small vessel disease is a neurological disease that affects over 90\% of elderly and is a major cause of stroke, dementia and cognitive decline. Biomarkers of the disease are visible in \textsc{mri} and are manually identified by trained clinicians. Lacunes are one such biomarker and are of particular interest for their potential role in the advancement of cerebral small vessel disease and in the incidence of stroke. However, the manual identification of lacunes is difficult, laborious, and largely inconsistent.

In this thesis, we present a convolutional neural networks model for the automated detection of lacunes in \textsc{mri}. The model is adapted from an existing model by \cite{GhafoorianM.2017Dml3}, whilst making vast simplifications. In particular, we perform extraction of the brain matter prior to modelling and remove the dependence on location-based variables. The resulting model exhibits a testing sensitivity of 99.9\% and a specificity of 99.8\%.

%\afterpage{\blankpage}
