%\documentclass[honours,12pt,twoside]{unswthesis}

\usepackage{afterpage}
\usepackage{amsfonts}
\usepackage{amsmath}
\usepackage{amssymb}
\usepackage{amsthm}
\usepackage[english]{babel}
\usepackage{graphicx}
\usepackage{natbib}
\usepackage[utf8]{inputenc}
\usepackage{latexsym}
\usepackage{url}
\usepackage{todonotes}
\usepackage{tikz}
\usepackage{pdfpages}
\usetikzlibrary{arrows}
\usepackage{float}
\usepackage[algoruled,boxed,lined]{algorithm2e}

\usepackage{booktabs}
\renewcommand{\arraystretch}{1.2}


%%%%%%%%%%%%%%%%%%%%%%%%%%%%%%%%%%%%%%%%%%%%%%%%%%%%%%%%%%%%%%%%%
%
%  The following are some simple LaTeX macros to give some
%  commonly used letters in funny fonts. You may need more or less of
%  these
%
\newcommand{\R}{\mathbb{R}}
\newcommand{\Q}{\mathbb{Q}}
\newcommand{\C}{\mathbb{C}}
\newcommand{\N}{\mathbb{N}}
\newcommand{\F}{\mathbb{F}}
\newcommand{\PP}{\mathbb{P}}
\newcommand{\T}{\mathbb{T}}
\newcommand{\Z}{\mathbb{Z}}
\newcommand{\B}{\mathfrak{B}}
\newcommand{\BB}{\mathcal{B}}
\newcommand{\M}{\mathfrak{M}}
\newcommand{\X}{\mathfrak{X}}
\newcommand{\Y}{\mathfrak{Y}}
\newcommand{\CC}{\mathcal{C}}
\newcommand{\E}{\mathbb{E}}
\newcommand{\cP}{\mathcal{P}}
\newcommand{\cS}{\mathcal{S}}
\newcommand{\A}{\mathcal{A}}
\newcommand{\ZZ}{\mathcal{Z}}

%%%%%%%%%%%%%%%%%%%%%%%%%%%%%%%%%%%%%%%%%%%%%%%%%%%%%%%%%%%%%%%%%%%%%
%
% The following are much more esoteric commands that I have left in
% so that this file still processes. Use or delete as you see fit
%
\newcommand{\bv}[1]{\mbox{BV($#1$)}}
\newcommand{\comb}[2]{\left(\!\!\!\begin{array}{c}#1\\#2\end{array}\!\!\!\right)
}
\newcommand{\Lat}{{\rm Lat}}
\newcommand{\var}{\mathop{\rm var}}
\newcommand{\Pt}{{\mathcal P}}
\def\tr(#1){{\rm trace}(#1)}
\def\Exp(#1){{\mathbb E}(#1)}
\def\Exps(#1){{\mathbb E}\sparen(#1)}
\newcommand{\floor}[1]{\left\lfloor #1 \right\rfloor}
\newcommand{\ceil}[1]{\left\lceil #1 \right\rceil}
\newcommand{\hatt}[1]{\widehat #1}
\newcommand{\modeq}[3]{#1 \equiv #2 \,(\text{mod}\, #3)}
\newcommand{\rmod}{\,\mathrm{mod}\,}
\newcommand{\p}{\hphantom{+}}
\newcommand{\vect}[1]{\mbox{\boldmath $ #1 $}}
\newcommand{\reff}[2]{\ref{#1}.\ref{#2}}
\newcommand{\psum}[2]{\sum_{#1}^{#2}\!\!\!'\,\,}
\newcommand{\bin}[2]{\left( \begin{array}{@{}c@{}}
				#1 \\ #2
			\end{array}\right)	}
%
%  Macros - some of these are in plain TeX (gasp!)
%
\newcommand{\be}{($\beta$)}
\newcommand{\eqp}{\mathrel{{=}_p}}
\newcommand{\ltp}{\mathrel{{\prec}_p}}
\newcommand{\lep}{\mathrel{{\preceq}_p}}
\def\brack#1{\left \{ #1 \right \}}
\def\bul{$\bullet$\ }
\def\cl{{\rm cl}}
\let\del=\partial
\def\enditem{\par\smallskip\noindent}
\def\implies{\Rightarrow}
\def\inpr#1,#2{\t \hbox{\langle #1 , #2 \rangle} \t}
\def\ip<#1,#2>{\langle #1,#2 \rangle}
\def\lp{\ell^p}
\def\maxb#1{\max \brack{#1}}
\def\minb#1{\min \brack{#1}}
\def\mod#1{\left \vert #1 \right \vert}
\def\norm#1{\left \Vert #1 \right \Vert}
\def\paren(#1){\left( #1 \right)}
\def\qed{\hfill \hbox{$\Box$} \smallskip}
\def\sbrack#1{\Bigl \{ #1 \Bigr \} }
\def\ssbrack#1{ \{ #1 \} }
\def\smod#1{\Bigl \vert #1 \Bigr \vert}
\def\smmod#1{\bigl \vert #1 \bigr \vert}
\def\ssmod#1{\vert #1 \vert}
\def\sspmod#1{\vert\, #1 \, \vert}
\def\snorm#1{\Bigl \Vert #1 \Bigr \Vert}
\def\ssnorm#1{\Vert #1 \Vert}
\def\sparen(#1){\Bigl ( #1 \Bigr )}

\newcommand\blankpage{%
    \null
    \thispagestyle{empty}%
    \addtocounter{page}{-1}%
    \newpage}
    
%%%%%%%%%%%%%%%%%%%%%%%%%%%%%%%%%%%%%%%%%%%%%%%%%%%%%%%%%%%%%%
%
% These environments allow you to get nice numbered headings
%  for your Theorems, Definitions etc.  
%
%  Environments
%
%%%%%%%%%%%%%%%%%%%%%%%%%%%%%%%
\newtheorem{theorem}{Theorem}[section]
\newtheorem{lemma}[theorem]{Lemma}
\newtheorem{proposition}[theorem]{Proposition}
\newtheorem{corollary}[theorem]{Corollary}
\newtheorem{conjecture}[theorem]{Conjecture}
%\theoremstyle{definition}
\newtheorem{definition}[theorem]{Definition}
\newtheorem{example}{Example}
\newtheorem{remark}[theorem]{Remark}
\newtheorem{question}[theorem]{Question}
\newtheorem{notation}[theorem]{Notation}
\numberwithin{equation}{section}

%\begin{document}


\begin{appendices}
\chapter{\textsc{r} code for candidate lacune detection model}\label{app-r-code}

\begin{verbatim}
library(tensorflow)
library(crayon)

# Data placeholders
# Num of samples. 51x51 = 2601. x2 channels = 5202
x <- tf$placeholder(tf$float32, shape(NULL, 5202L))
# Num of samples, 2 outcomes
y_ <- tf$placeholder(tf$float32, shape(NULL, 2L))

# Weights initialised by He Method
he.init <- tf$contrib$layers$
  variance_scaling_initializer(factor = 2.0,
                                mode = "FAN_AVG",
                                uniform = FALSE)
# Variable initialiser functions
weight.variable <- function(shape) {
  initial <- he.init(shape)
  tf$Variable(initial)
}

beta.variable <- function(shape) {
  tf$Variable(tf$zeros(shape))
}

scale.variable <- function(shape) {
  tf$Variable(tf$ones(shape))
}

conv2d <- function(x, W) {
  tf$nn$conv2d(x, W, strides = c(1, 1, 1, 1), padding = "VALID")
}

# Batch Normlisation
batch.norm <- function(z, beta, scale) {
  moments <- tf$nn$moments(z, 0L)
  tf$nn$batch_normalization(z, moments[[1]], moments[[2]]^2,
                            beta,scale, 1e-3)
}


# Dropout of 0.3 on fully connected layers (avoids overfitting)
keep.prob <- tf$placeholder(tf$float32)


# Reshape x into sample size of 51x51
# 2 colour channels (FLAIR & T1)
x.image <- tf$reshape(x, shape(-1L, 51L, 51L, 2L))


# Conv Layers ---------------------------------------------------

# conv 1: 20 filters, 7x7 size
W.conv1 <- weight.variable(shape(7L, 7L, 2L, 20L))
z.conv1 <- conv2d(x.image, W.conv1)
# Batch normalisation
beta.conv1 <- beta.variable(shape(20L))
scale.conv1 <- scale.variable(shape(20L))
bn.conv1 <- batch.norm(z.conv1, beta.conv1, scale.conv1)
a.conv1 <- tf$nn$relu(bn.conv1)

# pool: 2x2 size, 2 stride
a.pool1 <- tf$nn$max_pool(a.conv1, ksize = c(1L,2L,2L,1L),
                          strides = c(1L,2L,2L,1L),
                          padding = "VALID")

# conv 2: 40 filters, 5x5 size
W.conv2 <- weight.variable(shape(5L, 5L, 20L, 40L))
z.conv2 <- conv2d(a.pool1, W.conv2)
beta.conv2 <- beta.variable(shape(40L))
scale.conv2 <- scale.variable(shape(40L))
bn.conv2 <- batch.norm(z.conv2, beta.conv2, scale.conv2)
a.conv2 <- tf$nn$relu(bn.conv2)

# conv 3: 80 filters, 3x3 size
W.conv3 <- weight.variable(shape(3L, 3L, 40L, 80L))
z.conv3 <- conv2d(a.conv2, W.conv3)
beta.conv3 <- beta.variable(shape(80L))
scale.conv3 <- scale.variable(shape(80L))
bn.conv3 <- batch.norm(z.conv3, beta.conv3, scale.conv3)
a.conv3 <- tf$nn$relu(bn.conv3)

# conv 4: 110 filters, 3x3 size
W.conv4 <- weight.variable(shape(3L, 3L, 80L, 110L))
z.conv4 <- conv2d(a.conv3, W.conv4)
beta.conv4 <- beta.variable(shape(110L))
scale.conv4 <- scale.variable(shape(110L))
bn.conv4 <- batch.norm(z.conv4, beta.conv4, scale.conv4)
a.conv4 <- tf$nn$relu(bn.conv4)


# Fully Connected Layers ----------------------------------------

a.conv4.flat <- tf$reshape(a.conv4, shape(-1L, 14L*14L*110L))
# full 1: 300 size
W.fcl1 <- weight.variable(shape(14L*14L*110L, 300L))
z.fcl1 <- tf$matmul(a.conv4.flat, W.fcl1)
beta.fcl1 <- beta.variable(shape(300L))
scale.fcl1 <- scale.variable(shape(300L))
bn.fcl1 <- batch.norm(z.fcl1, beta.fcl1, scale.fcl1)
a.fcl1 <- tf$nn$relu(bn.fcl1)
a.fcl1.drop <- tf$nn$dropout(a.fcl1, keep.prob)

# full 2: 200 size
W.fcl2 <- weight.variable(shape(300L, 200L))
z.fcl2 <- tf$matmul(a.fcl1.drop, W.fcl2)
beta.fcl2 <- beta.variable(shape(200L))
scale.fcl2 <- scale.variable(shape(200L))
bn.fcl2 <- batch.norm(z.fcl2, beta.fcl2, scale.fcl2)
a.fcl2 <- tf$nn$relu(bn.fcl2)
a.fcl2.drop <- tf$nn$dropout(a.fcl2, keep.prob)

# full 3: 2 size
W.fcl3 <- weight.variable(shape(200L, 2L))
z.fcl3 <- tf$matmul(a.fcl2.drop, W.fcl3)
beta.fcl3 <- beta.variable(shape(2L))
scale.fcl3 <- scale.variable(shape(2L))
bn.fcl3 <- batch.norm(z.fcl3, beta.fcl3, scale.fcl3)

# Softmax classifier
y <- tf$nn$softmax(bn.fcl3)


# Training ------------------------------------------------------

# Cross entropy loss
# L2 regularisation with lambda_2 = 0.0001
cross.entropy <- tf$reduce_mean(
  -tf$reduce_sum(y_ * tf$log(y + 1e-10),reduction_indices = 1L))
l2.reg <- cross.entropy + 0.0001 * tf$reduce_sum(W.fcl3^2)

# Stochastic gradient descent (Adam optimiser)
learn.rate <- tf$placeholder(tf$float32)
train.step <- tf$train$AdamOptimizer(learn.rate)$minimize(l2.reg)

# Calculate prediction accuracy
correct.prediction <- tf$equal(tf$argmax(y, 1L),
                               tf$argmax(y_, 1L))
accuracy <- tf$reduce_mean(tf$cast(correct.prediction,
                                   tf$float32))

# Model saver
saver <- tf$train$Saver()

# Tensorflow session
sess <- tf$InteractiveSession()
sess$run(tf$global_variables_initializer())

# Stochastic gradient descent: batches of 128
num.samples <- nrow(training)
max.epochs <- 40
# Decaying learning rate. 5e-4 reduced to 1e-6
learning.rates <- seq(5e-4, 1e-6, length.out = max.epochs)

# Store training and validation accuracies
train.accuracy <- numeric(max.epochs*num.samples)
i.train.acc <- 1
valid.accuracy <- numeric(max.epochs)
i.valid.acc <- 1

best.accuracy <- 0
e <- 1
# Model training loop
while (e < max.epochs) {
  stoch.training <- training[sample(nrow(training)),]
  for (i in seq(1, num.samples-128, by = 128)) {
    # Run training step
    train.step$run(feed_dict = dict(
      x = stoch.training[i:(i+127), 5:5206],
      y_ = stoch.training[i:(i+127), 5207:5208],
      keep.prob = 0.7, learn.rate = learning.rates[e]))
    
    # Report training accuracy every 5 batches
    if (i %% 5 == 0) {
      train.accuracy[i.train.acc] <- accuracy$eval(
        feed_dict = dict(
        x = training[i:(i+127), 5:5206],
        y_ = training[i:(i+127), 5207:5208],
        keep.prob = 1.0, learn.rate = learning.rates[e]))
      cat(sprintf(" Acc: %g", train.accuracy[i.train.acc]))
      i.train.acc <- i.train.acc + 1
      plot(train.accuracy[max(i.train.acc-500,1):i.train.acc])
    }
    cat("\n")
  }
  # Report validation accuracy
  n.valid <- 3000
  valid.accuracy[i.valid.acc] <- accuracy$eval(feed_dict = dict(
    x = validation[1:n.valid,5:5206],
    y_ = validation[1:n.valid,5207:5208],
    keep.prob = 1.0, learn.rate = learning.rates[e]))
  cat(sprintf("epoch %d, validation accuracy %g\n",
              e, valid.accuracy[i.valid.acc]))
  
  # Early stopping - highest accuracy on validation set
  if(valid.accuracy[i.valid.acc] > best.accuracy) {
    cat("Saving Model..\n")
    best.accuracy <- valid.accuracy[i.valid.acc]
    saver$save(sess, "/srv/scratch/z5016924/model.ckpt")
  }
  i.valid.acc <- i.valid.acc + 1
  
  e <- e + 1
}
\end{verbatim}

\end{appendices}




%%%%%%%%%%%%%%%%%%%%%%%%%%%%%%%%%%%%%%%%%%%%%%%%%%%%%%%%%%%%%%%%%%%%%%%%%%

%\clearpage

\addcontentsline{toc}{chapter}{References}

\bibliographystyle{plain}
\bibliography{bibliography}

\end{document}



\end{document}


