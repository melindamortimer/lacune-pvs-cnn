%\documentclass[honours,12pt,twoside]{unswthesis}

\usepackage{afterpage}
\usepackage{amsfonts}
\usepackage{amsmath}
\usepackage{amssymb}
\usepackage{amsthm}
\usepackage[english]{babel}
\usepackage{graphicx}
\usepackage{natbib}
\usepackage[utf8]{inputenc}
\usepackage{latexsym}
\usepackage{url}
\usepackage{todonotes}
\usepackage{tikz}
\usepackage{pdfpages}
\usetikzlibrary{arrows}
\usepackage{float}
\usepackage[algoruled,boxed,lined]{algorithm2e}

\usepackage{booktabs}
\renewcommand{\arraystretch}{1.2}


%%%%%%%%%%%%%%%%%%%%%%%%%%%%%%%%%%%%%%%%%%%%%%%%%%%%%%%%%%%%%%%%%
%
%  The following are some simple LaTeX macros to give some
%  commonly used letters in funny fonts. You may need more or less of
%  these
%
\newcommand{\R}{\mathbb{R}}
\newcommand{\Q}{\mathbb{Q}}
\newcommand{\C}{\mathbb{C}}
\newcommand{\N}{\mathbb{N}}
\newcommand{\F}{\mathbb{F}}
\newcommand{\PP}{\mathbb{P}}
\newcommand{\T}{\mathbb{T}}
\newcommand{\Z}{\mathbb{Z}}
\newcommand{\B}{\mathfrak{B}}
\newcommand{\BB}{\mathcal{B}}
\newcommand{\M}{\mathfrak{M}}
\newcommand{\X}{\mathfrak{X}}
\newcommand{\Y}{\mathfrak{Y}}
\newcommand{\CC}{\mathcal{C}}
\newcommand{\E}{\mathbb{E}}
\newcommand{\cP}{\mathcal{P}}
\newcommand{\cS}{\mathcal{S}}
\newcommand{\A}{\mathcal{A}}
\newcommand{\ZZ}{\mathcal{Z}}

%%%%%%%%%%%%%%%%%%%%%%%%%%%%%%%%%%%%%%%%%%%%%%%%%%%%%%%%%%%%%%%%%%%%%
%
% The following are much more esoteric commands that I have left in
% so that this file still processes. Use or delete as you see fit
%
\newcommand{\bv}[1]{\mbox{BV($#1$)}}
\newcommand{\comb}[2]{\left(\!\!\!\begin{array}{c}#1\\#2\end{array}\!\!\!\right)
}
\newcommand{\Lat}{{\rm Lat}}
\newcommand{\var}{\mathop{\rm var}}
\newcommand{\Pt}{{\mathcal P}}
\def\tr(#1){{\rm trace}(#1)}
\def\Exp(#1){{\mathbb E}(#1)}
\def\Exps(#1){{\mathbb E}\sparen(#1)}
\newcommand{\floor}[1]{\left\lfloor #1 \right\rfloor}
\newcommand{\ceil}[1]{\left\lceil #1 \right\rceil}
\newcommand{\hatt}[1]{\widehat #1}
\newcommand{\modeq}[3]{#1 \equiv #2 \,(\text{mod}\, #3)}
\newcommand{\rmod}{\,\mathrm{mod}\,}
\newcommand{\p}{\hphantom{+}}
\newcommand{\vect}[1]{\mbox{\boldmath $ #1 $}}
\newcommand{\reff}[2]{\ref{#1}.\ref{#2}}
\newcommand{\psum}[2]{\sum_{#1}^{#2}\!\!\!'\,\,}
\newcommand{\bin}[2]{\left( \begin{array}{@{}c@{}}
				#1 \\ #2
			\end{array}\right)	}
%
%  Macros - some of these are in plain TeX (gasp!)
%
\newcommand{\be}{($\beta$)}
\newcommand{\eqp}{\mathrel{{=}_p}}
\newcommand{\ltp}{\mathrel{{\prec}_p}}
\newcommand{\lep}{\mathrel{{\preceq}_p}}
\def\brack#1{\left \{ #1 \right \}}
\def\bul{$\bullet$\ }
\def\cl{{\rm cl}}
\let\del=\partial
\def\enditem{\par\smallskip\noindent}
\def\implies{\Rightarrow}
\def\inpr#1,#2{\t \hbox{\langle #1 , #2 \rangle} \t}
\def\ip<#1,#2>{\langle #1,#2 \rangle}
\def\lp{\ell^p}
\def\maxb#1{\max \brack{#1}}
\def\minb#1{\min \brack{#1}}
\def\mod#1{\left \vert #1 \right \vert}
\def\norm#1{\left \Vert #1 \right \Vert}
\def\paren(#1){\left( #1 \right)}
\def\qed{\hfill \hbox{$\Box$} \smallskip}
\def\sbrack#1{\Bigl \{ #1 \Bigr \} }
\def\ssbrack#1{ \{ #1 \} }
\def\smod#1{\Bigl \vert #1 \Bigr \vert}
\def\smmod#1{\bigl \vert #1 \bigr \vert}
\def\ssmod#1{\vert #1 \vert}
\def\sspmod#1{\vert\, #1 \, \vert}
\def\snorm#1{\Bigl \Vert #1 \Bigr \Vert}
\def\ssnorm#1{\Vert #1 \Vert}
\def\sparen(#1){\Bigl ( #1 \Bigr )}

\newcommand\blankpage{%
    \null
    \thispagestyle{empty}%
    \addtocounter{page}{-1}%
    \newpage}
    
%%%%%%%%%%%%%%%%%%%%%%%%%%%%%%%%%%%%%%%%%%%%%%%%%%%%%%%%%%%%%%
%
% These environments allow you to get nice numbered headings
%  for your Theorems, Definitions etc.  
%
%  Environments
%
%%%%%%%%%%%%%%%%%%%%%%%%%%%%%%%
\newtheorem{theorem}{Theorem}[section]
\newtheorem{lemma}[theorem]{Lemma}
\newtheorem{proposition}[theorem]{Proposition}
\newtheorem{corollary}[theorem]{Corollary}
\newtheorem{conjecture}[theorem]{Conjecture}
%\theoremstyle{definition}
\newtheorem{definition}[theorem]{Definition}
\newtheorem{example}{Example}
\newtheorem{remark}[theorem]{Remark}
\newtheorem{question}[theorem]{Question}
\newtheorem{notation}[theorem]{Notation}
\numberwithin{equation}{section}

%\begin{document}



\chapter{Neuroimaging background}\label{mri_svd_intro}

An overview of magnetic resonance imaging (\textsc{mri}) and small vessel disease (\textsc{svd}) is required to understand the motivations, data and the features that are being detected. This chapter introduces the general structure of the brain, \textsc{mri} and the resulting images, \textsc{svd} biomarkers, and existing rating standards.

\section{Structure and terminology}

\subsection*{\textsc{mri} axes}

By convention, the \textsc{mri} axis planes are referred to as axial, coronal, and sagittal, as shown in Figure \ref{svd-axes}. 2D image generated from the axial plane are called \textit{slices}.

\begin{figure}[ht]
	\centering
	\includegraphics[scale=0.8]{Images/2_axes.png}
	\caption{\textsc{mri} axis planes.}
	\small \textbf{Image from}
	\label{svd-axes}
\end{figure}

\todo[inline]{Image reference?}

\subsection*{Structure of the brain}\label{svd-brain}

% Cerebrum diagram
\begin{figure}[ht]
	\centering
	\includegraphics[width=0.7\textwidth]{Images/2_Lobes_of_the_brain_NL.png}
	\caption{The four lobes of the cerebral cortex.}
	\small Image taken from Wikimedia Commons: \url{`Gray728.svg'}
	\label{svd-cerebrumfig}
\end{figure}

Information from throughout the body is communicated via nerves through the spinal chord to the brain. The brain is the most complex organ in the body, tasked with receiving, interpreting, and responding to these signals. It can be segmented into a number of regions, each responsible for different roles.

The largest region is the \textit{cerebrum}, shown in Figure \ref{svd-cerebrumfig}, which forms the outer surface of the brain. It is responsible for voluntary actions, senses, thought and memory, and is divided into two hemispheres - left and right. Each hemisphere is divided into four lobes:
 \begin{itemize}
	\item The \textit{frontal lobe}, located at the front of the cerebrum, is responsible for voluntary movement, skills and behaviours, mood, and memory.
	\item The \textit{parietal lobe}, situated posterior to the frontal lobe, is responsible for the senses, including pain, and physical and spatial awareness.
	\item The \textit{temporal lobe}, located exterior to the perietal lobe, is responsible for memory and auditory functions, including hearing and speech.
	\item The \textit{occipital lobe}, located posterior to the parietal lobe, is responsible for visual information.
\end{itemize}

% Gray vs white matter diagram
\begin{figure}[ht]
	\centering
	\includegraphics[width=0.7\textwidth]{Images/2_white_vs_grey.png}
	\caption{Grey matter occurs at the surface and within central structures such as the spinal chord. The white matter connects these structures together.}
	\small Image taken from \url{`https://medlineplus.gov/ency/imagepages/18117.htm'}
	\label{svd-greywhitefig}
\end{figure}


% Basal Ganglia
\begin{figure}[ht]
	\centering
	\includegraphics[width=0.7\textwidth]{Images/2_Basal_Ganglia_and_Related_Structures.png}
	\caption{The basal ganglia and other related structures.}
	\small Image taken from Wikimedia Commons: \url{`Basal_Ganglia_and_Related_Structures.svg'}
	\label{svd-basalfig}
\end{figure}


The outer surface of the brain consists of a layer of neurons referred to as \textit{grey matter}. This layer has a thickness of around 4 mm, where much of the brain processes occur. Grey matter also occurs in the cerebellum, brainstem, and within the spinal chord.

Underneath the grey matter is a network of fibres that connects these grey matter neurons together. Collectively, they form the \textit{white matter}. The grey and white matter are shown in Figure \ref{svd-greywhitefig}.

At the centre of the brain is a group of structures that form the \textit{basal ganglia}, shown in Figure \ref{svd-basalfig}. This region of the brain is responsible for voluntary movement and learning. Connected to this structure is the thalamus, which is also related to sensory and motor function.
%; and exhibits more numerous instances of lacunes and perivascular spaces. Two structures within the basal ganglia that are often found to have lacunes include the caudate and putamen. The thalamus is another structure that has a high frequency of lacunes, and is interconnected to the basal ganglia.

At the base of the brain lies the cerebellum and brain stem, also shown in Figure \ref{svd-basalfig}. They are responsible for coordination and the transmission of nerve communications respectively.

Within the skull, the brain sits in \textit{cerebral spinal fluid} (\textsc{csf}). This fluid can flow between the ridges  at the surface of the brain, filling gaps throughout the brain matter.


\section{\textsc{mri}}\label{svd-MRI}

(\textsc{mri}) is a radiological technique that uses magnetic fields and radio waves to generate greyscale images of organs inside the body. The two most common images produced are \textit{T1-weighted} and \textit{T2-weighted} images, shown in Figure \ref{svd-t1-vs-t2}.

% Image comparisons
\begin{figure}[ht]
	\centering
	\includegraphics[width=\textwidth]{Images/2_t1_t2_flair.jpg}
	\caption{A comparison of T1, T2 and \textsc{flair} images.}
	\small Image taken from \cite{Preston2006}
	\label{svd-t1-vs-t2}
\end{figure}

T1-weighted images are bright in regions with high fat content, such as white matter, and are dark in regions with high water content, usually \textsc{csf}. In contrast, T2-weighted images are bright in regions with high fat contents and water. This can make it easier to spot abnormalities.

\textit{FLuid-Attenuated Inversion Recovery} (\textsc{flair}) is another commonly used sequence in \textsc{mri} rating \cite{WardlawJ.M.2013Nsfr}. It is similar to T2-weighted imaging, except that \textsc{csf} remains dark. This allows for easier identification of abnormalities amongst \textsc{csf}.

% T1, T2 and FLAIR images

\textsc{mri} scan data is composed of numerous two-dimensional slices, that together form a three-dimensional volume. The resolution of a \textsc{mri} scan is higher than that of other scans, such as a \textsc{ct} scan. Hence \textsc{mri} scanning is frequently preferred for neurological and cancer studies.

\section{\textsc{svd} biomarkers}\label{svd-markers}

In the analysis of \textsc{mri} scans for \textsc{svd}, there are a number of biomarkers that clinicians observe. Each of these is defined in conjunction with the \textsc{strive} criterion \cite{WardlawJ.M.2013Nsfr}.

\textit{White matter hyperintensities} (\textsc{wmh}) are regions of hyperintensity visible in T2-weighted imaging. They also appear in T1-weighted images as hypointense regions, though not as dark as \textsc{csf}. Their cause is not well understood.

\textit{Lacunes} are small brain cavities. They usually appear without symptoms, and are frequently found in the scans of elderly. Their presence indicates a heightened risk of stroke and dementia \cite{BenjaminJ.Philip2018LIbN,VanDerFlierM.Wiesje2005SVDa}. In \textsc{mri}, lacunes appear round, with a diameter between 3 mm and 15 mm. As they are filled with fluid, they tend to give off a darker signal intensity, similar to that of \textsc{csf}. Lacunes have a tendency to occur in regions of white matter hyperintensity, so will frequently have a hyperintense rim in \textsc{flair} imaging.

\textit{Perivascular spaces} are extensions of the fluid space surrounding blood vessels through the brain. They are generally microscopic but can become enlarged with age, and often appear alongside other \textsc{svd} biomarkers such as lacunes and \textsc{wmh}. As they are fluid-filled, perivascular spaces also take on a similar signal intensity to \textsc{csf}. They are found running parallel to vessels, and are generally found under 3 mm in diameter. As they are tubular, they can be identified by appearing circular cross-sectionally, but oblong when viewed in parallel to the vessels.

In some instances, perivascular spaces can become enlarged, found up to 10 mm in diameter. As their signal intensity is similar to that of lacunes, they can be difficult to distinguish. 

% Images of lacunes and perivascular spaces from \textsc{strive}
\begin{figure}[ht]
	\centering
	\includegraphics[width = \textwidth]{Images/2_STRIVE.png}
	\caption{\textsc{strive} criterion and \textsc{mri} examples.}
	\small Image taken from \cite{WardlawJ.M.2013Nsfr}
\end{figure}

Other biomarkers of interest include cerebral microbleeds, recent small subcortical infarcts and brain atrophy. 

\textit{Cerebral microbleeds} are blooming regions of microscopic bleeding, around 2-5 mm in diameter, though they can be larger. They are not visible on T1-weighted, T2-weighted or \textsc{flair} images, and are instead found in T2*-imaging, constructed as a combination of T2-imaging and inhomogeneities in the magnetic field during scanning. 

\textit{Recent Small subcortical infarcts}, also called \textit{lacunar strokes}, are the cause of 25\% of ischaemic (oxygen starved) strokes \cite{WardlawJ.M.2013Nsfr}. They are regions of recent oxygen deprivation that have resulted in dying cells. These lesions are usually less than 20 mm in diameter. 

\textit{Brain atrophy} refers to the reduction of brain matter and is not restricted to particular regions of the brain. It can be identified by the increase in \textsc{csf} volume in T1-weighted, T2-weighted and \textsc{flair} imaging.

\section{Image rating}\label{svd-rating}

Without a biopsy for confirmation, the identification of \textsc{svd} biomarkers relies on \textsc{mri} analysis. Trained observers examine \textsc{mri} slice by slice and identify any lesions or points of interest.

Prior to 2013, there were no official guidelines for the identification of \textsc{svd} biomarkers. There were several studies attempting to establish rating guidelines \cite{AdamsH.H.Hieab2013RMfD, PotterGillian2015CPSV}, however these methods tended to focus on specific events rather than \textsc{svd} biomarkers in general.

%In addition, much of the terminology surrounding some biomarkers was inconsistent. For instance, perivascular spaces are also frequently referred to as Virchow-Robin spaces \cite{AdamsH.H.Hieab2013RMfD, WardlawJ.M.2013Nsfr}.

In 2013, the \textsc{strive} criterion \cite{WardlawJ.M.2013Nsfr} were established to standardise the terminology and definitions. Visual rating was then conducted in conjunction with those guidelines. Though the criterion helped to improve rating consistency, the appearance of lacunes and perivascular spaces are highly similar and therefore remain difficult to distinguish. As a result, manual rating is still highly inconsistent \cite{PotterGillian2015CPSV}. 

In addition, the manual rating process is also time consuming. The checking and logging of an individual scan can take over 10 minutes.

It is only recently that machine learning algorithms have begun to improve the rating process. Dou et al. \cite{DouQ.2016ADoC} released a machine learning algorithm for the detection of cerebral microbleeds. This algorithm exhibited a sensitivity of 93.16\%, with an average of 2.74 false positives per slice. 

Ghafoorian et al. \cite{GhafoorianM.2017Dml3} developed a machine learning algorithm for the automated detection of lacunes. This algorithm was able to achieve a sensitivity of 97.4\%, with 0.13 false positives per slice. This algorithm will be discussed further in Section \ref{litrev-ghafoorian}.

%%%%%%%%%%%%%%%%%%%%%%%%%%%%%%%%%%%%%%%%%%%%%%%%%%%%%%%%%%%%%%%%%%%%%%%%%%

%\clearpage

\addcontentsline{toc}{chapter}{References}

\bibliographystyle{plain}
\bibliography{bibliography}

\end{document}



\end{document}

